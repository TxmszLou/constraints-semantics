\documentclass[xcolor=dvipsnames]{beamer}

\usepackage{notation/modes}

% \usetheme{Frankfurt}
\usetheme{Madrid}

\makeatother
\setbeamertemplate{footline}
{
  \leavevmode%
  \hbox{%
  \begin{beamercolorbox}[wd=.4\paperwidth,ht=2.25ex,dp=1ex,center]{author in head/foot}%
    \usebeamerfont{author in head/foot}\insertshortauthor
  \end{beamercolorbox}%
  \begin{beamercolorbox}[wd=.6\paperwidth,ht=2.25ex,dp=1ex,center]{title in head/foot}%
    \usebeamerfont{title in head/foot}\insertshorttitle\hspace*{3em}
    \insertframenumber{} / \inserttotalframenumber\hspace*{1ex}
  \end{beamercolorbox}}%
  \vskip0pt%
}
\makeatletter
\setbeamertemplate{navigation symbols}{}


\begin{document}

\title{Constraints as a Denotational Semantics for Object Calculus}
\subtitle{Midterm presentation}
\author{Thomas Sixuan Lou}

\frame{\titlepage}

\section{Project Overview}

\begin{frame}
  \frametitle{Project Overview \& Motivations}
  Motivation: We want to solve \emph{local equations} and \emph{local behaviors}
  over arbitrary algebras.

  \bigskip

  \pause

  \begin{itemize}
    \item local equations (constraints) $\sim$ classes.
      \pause
    \item local solutions $\sim$ objects.
      \pause
    \item constraint resolution $\sim$ semantics for object calculus.
  \end{itemize}

\end{frame}

\begin{frame}
  \frametitle{Object Calculus in a Nutshell}

  \begin{itemize}
    \item Values (``objects''/``instances'') of a type are classified into ``classes''.
      \pause
    \item Functions (``methods'') have different behaviors depending on the ``class'' of its arguments.
      \pause
    \item Each ``class'' enforces different internal structures and different behaviors when passed to methods.
      \pause
    \item We represent each class by a record of field and function types.
      \pause
    \item We represent the behavior of each method with a ``dispatch matrix''
      \[ \prod_{c \in \mathcal{C}} \prod_{d \in \mathcal{D}} ( \IMode{\tau^c} \to \OMode{\rho_d}) \]
  \end{itemize}

\end{frame}

\begin{frame}
  \frametitle{Object Calculus: examples}
  
\end{frame}

\begin{frame}
  \frametitle{Adding Constraints in Classes}

  Now, we allow the declaration of \emph{local equations} in objects.\\

  \pause

  \[
    Class \ \{ s \hookrightarrow A , t \hookrightarrow B , s = t \}
  \]

  \pause

  An instance solves constraints declared by its class.

  \pause

  \[
    Obj \ \{ s \mapsto a , t \mapsto b \} : Class \ \{ s \hookrightarrow A , t \hookrightarrow B , s = t \}
  \]

  \pause

  and also first-order constraints!

  \pause

  \[
    Class \ \{ &  s = t , \neg \phi , \phi_1 \land \phi_2 , \phi_1 \vee \phi_2 , \exists x . \phi \}\\
  \]

  % Classes have equality constraints between terms, and an object of that class
  % contains data what values each variable maps to, and they should satisfy
  % constraints declared in the class.

\end{frame}

\begin{frame}
  \frametitle{Object Calculus with Constraints: examples}
  
\end{frame}

\begin{frame}
  \frametitle{Problem Statement}
  
\end{frame}

\begin{frame}
  \frametitle{The Calculus}
  
\end{frame}

\begin{frame}
  \frametitle{Semantics of Primitive Constraints}
  
\end{frame}

\begin{frame}
  \frametitle{Semantics of the Calculus}
  
\end{frame}

\end{document}