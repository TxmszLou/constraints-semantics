\documentclass[xcolor=dvipsnames]{beamer}

\usepackage{notation/modes}
\usepackage{stmaryrd}

% \usetheme{Frankfurt}
\usetheme{Madrid}

\makeatother
\setbeamertemplate{footline}
{
  \leavevmode%
  \hbox{%
  \begin{beamercolorbox}[wd=.4\paperwidth,ht=2.25ex,dp=1ex,center]{author in head/foot}%
    \usebeamerfont{author in head/foot}\insertshortauthor
  \end{beamercolorbox}%
  \begin{beamercolorbox}[wd=.6\paperwidth,ht=2.25ex,dp=1ex,center]{title in head/foot}%
    \usebeamerfont{title in head/foot}\insertshorttitle\hspace*{3em}
    \insertframenumber{} / \inserttotalframenumber\hspace*{1ex}
  \end{beamercolorbox}}%
  \vskip0pt%
}
\makeatletter
\setbeamertemplate{navigation symbols}{}


\begin{document}

\title{Constraints as a Denotational Semantics for Object Calculus}
\subtitle{Midterm presentation}
\author{Thomas Sixuan Lou}

\frame{\titlepage}

\section{Project Overview}

\begin{frame}
  \frametitle{Project Overview \& Motivations}
  Motivation: We want to solve \emph{local equations} and \emph{local behaviors}
  over arbitrary algebras.

  \bigskip

  \pause

  \begin{itemize}
    \item local equations (constraints) $\sim$ classes.
      \pause
    \item local solutions $\sim$ objects.
      \pause
    \item constraint resolution $\sim$ semantics for object calculus.
  \end{itemize}

\end{frame}

\begin{frame}
  \frametitle{Object Calculus in a Nutshell}

  \begin{itemize}
    \item Values (``objects''/``instances'') of a type are classified into ``classes''.
      \pause
    \item Functions (``methods'') have different behaviors depending on the ``class'' of its arguments.
      \pause
    \item Each ``class'' enforces different internal structures and different behaviors when passed to methods.
      \pause
    \item We represent each class by a record of field and function types.
      \pause
    \item We represent the behavior of each method with a ``dispatch matrix''
      \[ \prod_{c \in \mathcal{C}} \prod_{d \in \mathcal{D}} ( \IMode{\tau^c} \rightharpoonup \OMode{\rho_d}) \]
  \end{itemize}

\end{frame}

\begin{frame}
  \frametitle{Object Calculus: examples}

  \begin{align*}
    \onslide<1->{
    \tau^{cart} & = Class \ \{ x : \mathbb{R} , y : \mathbb{R} \}\\
    \tau^{polar} & = Class \ \{ r : \mathbb{R} , \theta : \mathbb{R} \}}\\
    \onslide<2->{
    p  &= Obj \ \{ x \mapsto 1 , y \mapsto 1\}  : \tau^{cart}\\
    q  &= Obj \ \{ r \mapsto 1 , \theta \mapsto \frac{\pi}{4}\}  : \tau^{polar}}\\
    \onslide<3->{
    e & :
    \begin{bmatrix}
      \tau^{cart} \to \rho_{distSq} & \tau^{cart} \to \rho_{xCoord}\\
      \tau^{polar} \to \rho_{distSq} & \tau^{polar} \to \rho_{xCoord}
    \end{bmatrix}\\
    e & =
    \begin{bmatrix}
      \lambda p : \tau^{cart} . (p.x^2 + p.y^2) & \lambda p : \tau^{cart} . p.x\\
      \lambda p : \tau^{polar} . (p.r^2)  & \lambda p : \tau^{polar} . (p.r \times \cos(p.\theta))
    \end{bmatrix}}
  \end{align*}

\end{frame}

\begin{frame}
  \frametitle{Adding Constraints in Classes}

  Now, we allow the declaration of \emph{local equations} in objects.\\

  \pause

  \[
    Class \ \{ s : A , t : B , s = t \}
  \]

  \pause

  An instance solves constraints declared by its class.

  \pause

  \[
    Obj \ \{ s \mapsto a , t \mapsto b \} : Class \ \{ s : A , t : B , s = t \}
  \]

  \pause

  and also first-order constraints!

  \pause

  \[
    Class \ \{ s = t , \neg \phi , \phi_1 \land \phi_2 , \phi_1 \vee \phi_2 , \exists x . \phi \}
  \]

  % Classes have equality constraints between terms, and an object of that class
  % contains data what values each variable maps to, and they should satisfy
  % constraints declared in the class.

\end{frame}

\begin{frame}
  \frametitle{Example: classes as problems}

  An arithmetic problem class:

  \begin{gather*}
    \onslide<1->{Class \ \{ x : \mathbb{N}, y : \mathbb{N} , (3 + x) \cdot 2 = y \land y = 2 \cdot 3 + 2 \}}\\
    \onslide<2->{\text{evaluate!}\\\Downarrow\\
    \llbracket Class \ \{ x : \mathbb{N}, y : \mathbb{N} , (3 + x) \cdot 2 = y \land
    y = 2 \cdot 3 + 2 \} \rrbracket_{AE}}\\
    \onslide<3->{\Downarrow\\
    Class \ \{ x : \mathbb{N}, y : \mathbb{N} , \llbracket (3 + x) \cdot 2 = y
    \rrbracket_{AE} \land \llbracket y = 2 \cdot 3 + 2 \rrbracket_{AE} \}}\\
    \onslide<4->{\Downarrow\\
    Class \ \{ x : \mathbb{N}, y : \mathbb{N} , (3 + x) \cdot 2 = y \land y = 8 \}}
  \end{gather*}
  
\end{frame}

\begin{frame}
  \frametitle{Example: objects as solutions}

  Possible solutions to a problem given by the semantics of equations

  \[
    Obj \ \{ x \mapsto 1 , y \mapsto 8\} : C
  \]

  We denote the ``solving'' process into the CLP land:\\
  let $\theta = \{x \mapsto 1 , y \mapsto 8\}$
  \begin{gather*}
    \onslide<1->{
    \llbracket Class \ \{ x : \mathbb{N}, y : \mathbb{N} , (3 + x) \cdot 2 = y \land
    y = 8 \} \rrbracket(\theta)}\\
    \onslide<2->{
    \Downarrow\\
    \OMode{8 = \llbracket (3 + x) \cdot 2 \rrbracket\theta \equiv \llbracket y
    \rrbracket \theta}\\
    \OMode{8 = \llbracket y \rrbracket \theta = \llbracket 8 \rrbracket \theta}}\\
    \onslide<3->{
    \Downarrow\\
    \{\theta\}}
  \end{gather*}
  
\end{frame}

\begin{frame}
  \frametitle{OO : inheritance}

  As in ordinary object calculus, there are inheritance to create class
  hierarchies.

  \begin{align*}
    \tau^{cart} & = Class \ \{ x : \mathbb{R} , y : \mathbb{R} \}\\
    \tilde{\tau}^{cart} & <: \tau^{cart} = Class \ \{ color : colorT, \color{gray} x : \mathbb{R} , y : \mathbb{R}\}
  \end{align*}

  same for constraints

  \begin{align*}
    \tilde{\tau}^{cart} & <: \tau^{cart} = Class \ \{ x + y = 1 , \color{gray} x : \mathbb{R} , y : \mathbb{R} \}\\
    \hat{\tau}^{cart}   & <: \tilde{tau}^{cart} = Class \ \{ x = 1  \color{gray} \land x + y = 1, x : \mathbb{R} , y : \mathbb{R} \}
  \end{align*}
  
  
\end{frame}

\begin{frame}
  \frametitle{OO : methods}

  Recall methods are (internally) represented as a dispatch matrix, when adding
  method to a class, we are adding a column vector into the class.

  \begin{align*}
    \tilde{\tau}^{cart} & <: \tau^{cart} = Class \ \{ dist : \prod_{c \in C} \tau^c \to \mathbb{R} \}\\
    \hat{\tau}^{cart} & <: \tilde{\tau}^{cart} = Class \ \{ dist(self) = 1\}
  \end{align*}

  \pause

  \[
    Obj \{ x \mapsto 1 , y \mapsto 0\} : \hat{\tau}^{cart}
  \]

\end{frame}

\begin{frame}
  \frametitle{Dispatch methods}

  Let $\theta = \{x \mapsto 1 , y \mapsto 0\}$,

  \begin{gather*}
    \llbracket Class \{ x : \mathbb{R} , y : \mathbb{R} , dist : \prod_{c \in
      \mathcal{C}} \tau^c \to \mathbb{R}\} , dist(self) = 1  \rrbracket\theta\\
    \Downarrow\\
    \OMode{\llbracket dist(self) \rrbracket\theta \stackrel{\text{dispatch}}{=}
      \llbracket (\lambda p : \tau^{cart} . (p.x^2 + p.y^2)) self
      \rrbracket\theta}\\
    \OMode{= \llbracket x^2 + y^2 \rrbracket\theta = 1 = \llbracket 1
      \rrbracket\theta}\\
    \Downarrow\\
    \{ \theta \}
  \end{gather*}
  
\end{frame}


% \begin{frame}
%   \frametitle{The Calculus}
  
% \end{frame}

% \begin{frame}
%   \frametitle{Semantics of Primitive Constraints}
  
% \end{frame}

% \begin{frame}
%   \frametitle{Semantics of the Calculus}
  
% \end{frame}

\end{document}