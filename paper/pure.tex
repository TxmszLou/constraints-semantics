\section{A Pure Calculus with Constraints}

In this section, we will define $\textbf{O}_c$, an pure object-based calculus with
constraints introduced through a special binder $\iota(o).e$.\\

The calculus is parametrized over an algebra and an theory on that algebra.
Intuitively, the algebra describes the meanings of those primitive syntactic
expressions. An theory defines a equivalence relation on that algebra, which the
solver will solve constraints with respect to.
% providing the
% meaning of the relation $\sim$
% For example, if we have an instance of the theory parametrized by the following arithmetic algebra:

\begin{exa}[Modular arithmetic]
  The arithmetic algebra $AE = \langle \mathbb{N} , 0, 1, +, \times \rangle$ is
  an algebra over with carriers of natural numbers $\mathbb{N}$, characteristic
  elements $0,1$ and the following binary operations:
  \begin{align*}
    +      : \mathbb{N} \to \mathbb{N} \to \mathbb{N}\\
    \times : \mathbb{N} \to \mathbb{N} \to \mathbb{N}
  \end{align*}
  satisfying
  \begin{alignat*}{2}
      0 + x \doteq x \doteq x + 0 \quad && x + (y + z) \doteq (x + y) + z\\
      1 \times x \doteq x \doteq x \times 1 \quad && x \times (y \times z) \doteq (x \times y) \times z
  \end{alignat*}

  $\mathbb{N}_m$ is an theory on $AE$ with the following equivalence relation:
  \[
    a \sim b \text{ iff } a \equiv b \pmod{m}
  \]

  % $\mathbb{N} /_\leq$ is a theory on $AE$ with:
  % \[
  %   a \sim b \text{ iff } a \leq b
  % \]
\end{exa}

\begin{exa}[Boolean algebra]
  A distributive lattice $\mathcal{L}$ on set $L$ is the algebra $\langle L,
  \land, \vee \rangle$ with operations $\land, \vee : L \to L \to L$ satisfying
  \begin{gather*}
    \begin{aligned}
      x \land x & \doteq x\\
      x \land y & \doteq y \land x\\
      x \vee y  & \doteq y \vee x\\
      x \land (y \land z) & \doteq (x \land y) \land z\\
      x \vee (y \vee z) & \doteq (x \vee y) \vee z
    \end{aligned}
    \qquad
    \begin{aligned}
      x \vee x & \doteq x\\
      x \vee (x \land y) & \doteq x\\
      x \land (x \vee y) & \doteq x\\
      x \land (y \vee z) & \doteq (x \land y) \vee (x \land z)\\
      x \vee (y \land z) & \doteq (x \vee y) \land (x \vee z)
    \end{aligned}
  \end{gather*}

  A boolean algebra $\mathcal{B} = \langle B, 0, 1, \land, \vee, ' \rangle$ is an
  extension to $\mathcal{L}$ with characteristic elements $0,1$ and an complement
  operator $' : B \to B$, satisfying
  \begin{equation*}
    \begin{aligned}
      x \land 0 \doteq 0\\
      x \vee 1 \doteq 1\\
      x' \land x \doteq 0
    \end{aligned}
    \qquad
    \begin{aligned}
      x \land 1 \doteq x\\
      x \vee 0 \doteq x\\
      x' \vee x \doteq 1
    \end{aligned}
  \end{equation*}

  $\textbf{Su}(X)$ (subsets of $X$) is a theory on $\mathcal{B}$ with
  the following equivalence relation:
  \[
    x \sim y \text{ iff } \forall z . z \in x \leftrightarrow z \in y
  \]
\end{exa}

In this paper, the constraint solver is abstracted as a blackbox that is able to
find a single best solution from a constraint whenever possible. The solver is
also parametrized over an algebra and a theory so that it's general enough to
solve equivalence constraints of arbitrary interpretations.\\

% \begin{defn}[Constraint]
  
% \end{defn}

For example, an solver $\mathfrak{S}$ over $\mathbb{N}/_{7}$ is expected to
be able to obtain a solution from constraints on the relation $\pmod{7}$: given
constraint $c = a \sim 0 \land b \sim a$

\[
  c \stackrel{\mathfrak{S}}{\rightsquigarrow} \{ a \mapsto 7 , b \mapsto 0\}
\]

\subsection{Syntax Formalization}

We now formally introduce the calculus. The syntax of the calculus is
constructed from a language $\mathcal{L}$, and various special syntactical
constructs. There are two syntactical categories, the object fragment
($a$) and the constraint fragment ($c,e$). They are separated so that rules
involving constraint solving are easier to write.\\

The complete syntax definition is given in \Cref{Oc:syntax}. Each object is
declared as a collection of $\langle \text{field name} , \text{method} \rangle$
and $\langle \text{constraint name} , \text{constraint} \rangle$ pairs. Each
method (resp. constraint) is given as a binder ``$\varsigma(x).a$'' (resp.
``$\iota(x).e$''), where $x$ is bound to which object it's invoked from.\\

We define free variables of $\textbf{O}_c$ terms and substitutions in
\Cref{Oc:fv} and \Cref{Oc:subst} respectively.

\begin{figure*}[t]
% \begin{mdframed}
  \centering
  \begin{alignat*}{3}
    \begin{aligned}
      c   & ::= \iota(x) . e && \text{constraint binder}\\
      e   & ::= && \text{constraint terms}\\
          & a_1 \sim a_2   && \quad \text{relation}\\
          & e_1 \land e_2  && \quad \text{conjunction}\\
          & e_1 \vee e_2   && \quad \text{disjunction}\\
          & \neg e_1       && \quad \text{negation}\\
          & \exists x . e_1 && \quad \text{decl of local variable}\\
     \mathcal{E}[-] & && \text{context}
    \end{aligned}
    \quad
    \begin{aligned}
      a,b & ::= && \text{terms}\\
          & t(a_1, \ldots, a_n)   && \quad \text{primitive terms } t \in \mathcal{L} \\
          & l,r  && \quad \text{labels}\\
          & x    && \quad \text{variable}\\
          & a.l_i \quad a.r_i && \quad \text{method/constraint invocation}\\
          & [ l_i = \varsigma (x) . b_i , r_j = c_j ]_{\substack{i = 1..n\\j = 1..m}} && \quad \text{object}\\
          & a.l_i \Leftarrow \varsigma (x) . b \quad a.r_i \Leftarrow c && \quad \text{method/constraint update}
    \end{aligned}
  \end{alignat*}
  \caption{Syntax of $\textbf{O}_c$ under theory $T$ on algebra $A$}
  \label{Oc:syntax}
% \end{mdframed}
\end{figure*}

\begin{figure*}[h]
  \centering
  \begin{alignat*}{2}
    \begin{aligned}
      FV(t(a_1 , \ldots , a_n)) & = \bigcup_{i \in [n]} FV(a_i)\\
      FV(x) & = \{x\}\\
      FV(l) & = \{l\}\\
      FV(a.l_i) & = FV(a)\\
      FV(a.r_i) & = FV(a)\\
      FV([l_i = \varsigma(x) . b_i , r_j = c_j]_{\substack{i \in [n]\\j \in [m]}})
      & = \bigcup_{i \in [n]} FV(\varsigma(x) . b_i) \cup \bigcup_{j \in [m]} FV(c_j)\\
      FV(a.l_i \Leftarrow \varsigma(x) . b) & = FV(a) \cup FV(\varsigma(x) . b)
    \end{aligned}
    \quad
    \begin{aligned}
      FV(a.r_i \Leftarrow c) & = FV(a) \cup FV(c)\\
      FV(\varsigma(x) . b) & = FV(b) \backslash \{x\}\\
      FV(\iota (x) . e) & = FV(e) \backslash \{x\}\\
      FV(a_1 \sim a_2) & = FV(a_1) \cup FV(a_2)\\
      FV(e_1 \land e_2) & = FV (e_1) \cup FV(e_2)\\
      FV(e_1 \vee e_2) & = FV (e_1) \cup FV(e_2)\\
      FV(\neg e) & = FV (e)\\
      FV(\exists x . e) & = FV(e) \backslash \{x\}
    \end{aligned}
  \end{alignat*}
  \caption{Free variables in $\textbf{O}_c$}
  \label{Oc:fv}
\end{figure*}

\begin{figure*}[h]
  \centering
  \begin{alignat*}{2}
    \begin{aligned}
      [a/y]t(a_1 , \ldots , a_n) & = t([a/y]a_1 , \ldots , [a/y]a_n)\\
      [a/y]x & = x\\
      [a/y]y & = a\\
      [a/y]b.l_i & = ([a/y]b).l_i\\
      [a/y]b.r_i & = ([a/y]b).r_i\\
      [a/y]\varsigma(x) . b & = \varsigma(z) [a/y][z/x]b\\
      [a/y]\iota(x) . b & = \iota(z) [a/y][z/x]b\\
      [a/y][l_i = \varsigma(x) . b_i , r_j = c_j] & = [l_i = [a/y]\varsigma(x) . b_i , r_j = [a/y]c_j]\\
    \end{aligned}
    \quad
    \begin{aligned}
      [a_1/y]a.l_i \Leftarrow \varsigma(x) . b & = ([a_1/y]a).l_i \Leftarrow [a_1/x]\varsigma(x). b\\
      [a_1/y]a.r_i \Leftarrow c & = ([a_1/y]a).r_i \Leftarrow [a_1/x]c\\
      [a/y]a_1 \sim a_2 & = [a/y]a_1 \sim [a/y]a_2\\
      [a/y]e_1 \land e_2 & = [a/y]e_1 \land [a/y]e_2\\
      [a/y]e_1 \vee e_2 & = [a/y]e_1 \vee [a/y]e_2\\
      [a/y]\neg e & = \neg ([a/y]e)\\
      [a/y]\exists x . e & = \exists z . ([a/y][z/x]e)\\
    \end{aligned}
  \end{alignat*}
  \caption{Substitution in $\textbf{O}_c$}
  \label{Oc:subst}
\end{figure*}


Since our calculus was parametrized over arbitrary algebras and theories, we
define the notion of evaluation that evaluate a term involving primitives (i.e.
with terms in $\mathcal{L}$) down to its general form by substituting primitives
with terms and formulas in the algebra.\\

In evaluating terms using the equational facts from the algebra, we have avoided
adding rules of evaluating arbitrary terms into our calculus. Thus, we only need
to focus the core object and constraint solving fragment in our calculus.

\begin{defn}[Evaluation]
  Let $\mathcal{L}$ be a language, let $\mathcal{A} = \langle A , \{f_n\}_n
  \rangle$ be an algebra and $\mathscr{I}$ be an interpretation of the language
  to the algebra. Let $f_{\mathscr{I}}$ denote the function $f$ in the language
  under interpretation $\mathscr{I}$ (it will be a function in $\mathcal{A}$). A
  \emph{evaluation} of term $a$ is defined inductively in \Cref{Oc:evaluation}.
  The evaluated term is called a \emph{generalized term}.
  \begin{figure*}[t]
    \centering
  \begin{align*}
    \begin{aligned}
      & \eval{f(a_1 , \ldots, a_2)} = f_{\mathscr{I}}(\eval{a_1} , \ldots , \eval{a_n})\\
      & \eval{x} = x\\
      & \eval{a.l_i} = \eval{a}.l_i\\
      & \eval{a.r_i} = \eval{a}.r_i\\
      & \eval{[l_i = \varsigma(x) . b_i , r_i = c_i]} =  [l_i = \varsigma(x) . \eval{b_i} , r_i = \eval{c_i}]\\
      & \eval{a.l_i \Leftarrow \varsigma (x) . b} = \eval{a}.l_i \Leftarrow \varsigma(x) . \eval{b}\\
      & \eval{a.r_i \Leftarrow c} = \eval{a}.l_i \Leftarrow \eval{c}
    \end{aligned}
        \quad
    \begin{aligned}
      & \eval{\iota(x) . e} = \iota(x) . \eval{e}\\
      & \eval{a_1 \sim a_2} = \iota(a_1) \sim \iota(a_2)\\
      & \eval{e_1 \land e_2} = \eval{e_1} \land \eval{e_2}\\
      & \eval{e_1 \vee e_2}   = \eval{e_1} \vee \eval{e_2}\\
      & \eval{\neg e}         = \neg \eval{e}\\
      & \eval{\exists x . e } = \exists x . \eval{e}\\
    \end{aligned}
  \end{align*}
    \caption{Evaluation}
    \label{Oc:evaluation}
  \end{figure*}

\end{defn}

\subsection{Semantics of Constraints}

Let $\solver$ be a solver parametrized by an algebra $\mathcal{A}$ and a theory
$\mathcal{T}$. It interacts with our calculus by the following satisfaction
judgment $o(c) \models_{\solver} S$, where $o$ is an object, $S$ is a set of
objects and $c$ is a constraint. The judgment means: $S$ is the set of objects that
satisfies constraint $c$ when evaluated from an particular object $o$.\\

This satisfaction judgment gives the semantics of equations (more generally,
$\sim$-relations) in our calculus. If we leave out the constraint $c$ and the
initial object $o$ is the judgment, it induces a semantic map on constraints:

\[
  \llbracket c \rrbracket : Object \to \mathcal{P}(Object)
\]

The link between the semantic map and the judgment is established by:

\[
  o(c) \models_{\solver} \llbracket c \rrbracket (o)
\]

\begin{figure*}[h]
  \centering
  \begin{gather*}
    \infer[\textsc{sat-eq}]
    {o(\iota(x) . a_1 \sim a_2) \models_{\solver} \{o\}}
    {[o/x]a_1 \rightsquigarrow v_1  &  [o/x]a_2 \to v_2  &  v_1 \sim v_2}\\
    \infer[\textsc{sat-empty}]
    {o(\iota(x) . a_1 \sim a_2) \models_{\solver} \varnothing}
    {[o/x]a_1 \in A  &  [o/x]a_2 \in A  &  [o/x]a_1 \rightsquigarrow v_1
       &  [o/x]a_2 \rightsquigarrow v_2  &  v_1 \not\sim v_2}\\
    \infer[\textsc{sat-update}_1]
    {o(\iota(x) . a_1 \sim a_2) \models_{\solver} \{o'\}}
    {l = [o/x]a_1  & [o/x]a_2 \rightsquigarrow v & l \not\in FV([o/x]a_2) & o.l \Leftarrow \varsigma(\rho(x)) v \to o'}\\
    \infer[\textsc{sat-update}_2]
    {o(\iota(x) . a_1 \sim a_2) \models_{\solver} \{o'\}}
    {\deduce{[o/x]a_1 = \hat{a}_1 \text{ is not a label or a variable}}{l =
        [o/x]a_2 & l \not\in FV(\hat{a}_1) & \hat{a}_1 \rightsquigarrow v} &
      o.l \Leftarrow \varsigma(\rho(x)) v \rightsquigarrow o'}\\
  \end{gather*}
  \caption{Satisfaction Judgment}
  \label{Oc:sat}
\end{figure*}

Thus, the judgment is an externalization of that semantic map, hiding away the
computational contents in interpreting constraints.\\

We may sometimes want to construct a new construct $c$ by combining several
constraints $\{c_i\}_{i \in I}$, aside from creating a new object with all of
$c_i$ inserted, we may also internalize it using first-order connectives we've
defined earlier.  Thus, it inspires us to define the following auxiliary
functions:

\begin{align*}
  (\iota(x) . e_1) \otimes (\iota(x) . e_2) \triangleq \iota(x) . e_1 \land e_2\\
  (\iota(x) . e_1) \oplus (\iota(x) . e_2) \triangleq \iota(x) . e_1 \vee e_2\\
\end{align*}


\subsubsection{Semantics of First-order constraints}

\begin{figure*}[h]
  \centering
  \caption{Satisfaction Judgment on First-order Constraints}
  \label{Oc:sat-foc}
\end{figure*}


\subsection{Reduction}


\begin{defn}[Reduction]
  
\end{defn}

\begin{figure*}[h]
  \centering
  \begin{gather*}
    \infer[\textsc{red-m-invoke}]{[l_i = \varsigma(x) . b_i , r_j = c_j].l_k
      \rightarrowtail [[l_i = \varsigma(x) . b_i , r_i = c_i]_{\substack{i \in [n]\\j \in [m]}} / x] b_k}{}\\
    \infer[\textsc{red-c-invoke}]
    {[l_i = \varsigma(x) . b_i , r_j = c_j].c_k \rightarrowtail o'}
    {[l_i = \varsigma(x) . b_i , r_j = c_j](c_k) \models_{\solver} S & o' \in S}\\
    \infer[\textsc{red-m-update}]
    {[l_i = \varsigma(x) . b_i , r_j = c_j].l_k \Leftarrow \varsigma(x) . b
      \rightarrowtail o'}
    {[l_k = \varsigma(x) . b , l_{i \neq k} = \varsigma(x) . b_i , r_j =
      c_j](c_1 \otimes \cdots \otimes c_m) \models_{\solver} S & o' \in S}\\
    \infer[\textsc{red-c-update}]
    {[l_i = \varsigma(x) . b_i , r_j = c_j].c_k \Leftarrow c \rightarrowtail o'}
    {[l_i = \varsigma(x) . b_i , r_{j \neq k} = c_j , r_k = c](c)
      \models_{\solver} S & o' \in S}\\
    \infer[\textsc{red-congr}]
    {\ctxt{a} \rightsquigarrow \ctxt{a'}}
    {a \rightarrowtail a'}
  \end{gather*}
  \caption{Reduction relation in $\textbf{O}_c$}
  \label{Oc:reduction}
\end{figure*}

\begin{thm}[monotonicity of objects]
  For any object $o = [l_i = \varsigma(x) . b_i , r_j = c_j]_{i \in [n], j \in [m]}$, given any
  constraint $c$, indices $p,q \in [m]$, if $o(c_p) \models_{\solver} S$ and $o.r_q
  \Leftarrow c \rightarrowtail o'$, then $o' \in S$.
\end{thm}


\subsection{Operational Semantics}