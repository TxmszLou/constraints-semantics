%% For double-blind review submission
% \documentclass[acmlarge,10pt,review,anonymous]{acmart}\settopmatter{printfolios=true}
%% For single-blind review submission
%\documentclass[sigplan,10pt,review]{acmart}\settopmatter{printfolios=true}
%% For final camera-ready submission
\documentclass[sigplan,10pt]{acmart}\settopmatter{}

%% Note: Authors migrating a paper from traditional SIGPLAN
%% proceedings format to PACMPL format should change 'sigplan,10pt' to
%% 'acmlarge'.


%% Some recommended packages.
\usepackage{amsmath,amssymb,amsthm,mathtools,paralist,bm,mdframed}
\usepackage{
  nameref, % \nameref
  hyperref, % \autoref
  cleveref, % \cref
}

\usepackage{mathrsfs}

% \usepackage{stmaryrd}

\usepackage{jms-proof}
\usepackage[french,english]{babel}
\usepackage[autostyle]{csquotes}

% \usepackage{tikz}
% \usetikzlibrary{cd}

% \usepackage{notation/modes}
% \newtheoremstyle{break}
%   {\topsep}{\topsep}%
%   {\itshape}{}%
%   {\bfseries}{}%
%   {\newline}{}%
% \theoremstyle{break}

\theoremstyle{plain}
\newtheorem{thm}{Theorem}[section]
\newtheorem{cor}[thm]{Corollary}
\newtheorem{lem}[thm]{Lemma}
\newtheorem{prob}[thm]{Problem}
\newtheorem{innercustomthm}{Problem}
\newenvironment{xprob}[1]
  {\renewcommand\theinnercustomthm{#1}\innercustomthm}
  {\endinnercustomthm}

\theoremstyle{definition}
\newtheorem{defn}[thm]{Definition}
\newtheorem{rem}[thm]{Remark}
\newtheorem{exa}[thm]{Example}

\numberwithin{equation}{section}
% \usepackage{unicode-math}
% \setmathfont{Latin Modern Math}
% \setmathfont[range={\mathscr,\mathbfscr}]{XITS Math}

% \newcommand{\comment}[1]{}
% \newcommand{\uncomment}[1]{#1}

\newenvironment{solu}{\begin{proof}[Solution]}{\end{proof}}

% notations
\DeclarePairedDelimiter\ceil{\lceil}{\rceil}
\DeclarePairedDelimiter\floor{\lfloor}{\rfloor}

\newcommand{\norm}[1]{\left\lVert#1\right\rVert}
\newcommand*{\mybox}[1]{\framebox{#1}}
\renewcommand\enquote[1]{\foreignquote{french}{#1}}

\newcommand\SetForm[2]{\left\{ #1 : #2 \right\}}
\newcommand{\eval}[1]{\mathfrak{e}({#1})}

\newcommand{\solver}{{\mathfrak{S}_{\mathcal{A},\mathcal{T}}}}
\newcommand{\solverP}[1]{{\mathfrak{S}_{#1}}}
\newcommand{\ctxt}[1]{{\mathcal{E}[{#1}]}}
\newcommand{\ectxt}{{\langle  \rangle}}

\DeclareMathAlphabet{\mathcal}{OMS}{cmsy}{m}{n}

% \oddsidemargin-1mm
% \evensidemargin-0mm
% \textwidth6.5in
% \topmargin-15mm
% %\headsep25pt
% \textheight8.75in
% \footskip27pt

\pagestyle{headings}

% \sloppy


%% Bibliography style
\bibliographystyle{ACM-Reference-Format}
%% Citation style
%% Note: author/year citations are required for papers published as an
%% issue of PACMPL.
\citestyle{acmauthoryear}  %% For author/year citations
%\citestyle{acmnumeric}     %% For numeric citations
%\setcitestyle{nosort}      %% With 'acmnumeric', to disable automatic
                            %% sorting of references within a single citation;
                            %% e.g., \cite{Smith99,Carpenter05,Baker12}
                            %% rendered as [14,5,2] rather than [2,5,14].
%\setcitesyle{nocompress}   %% With 'acmnumeric', to disable automatic
                            %% compression of sequential references within a
                            %% single citation;
                            %% e.g., \cite{Baker12,Baker14,Baker16}
                            %% rendered as [2,3,4] rather than [2-4].



\begin{document}

%% Title information
\title[Short Title]{Denotational Semantics of first-order Object Calculus}         %% [Short Title] is optional;
                                        %% when present, will be used in
                                        %% header instead of Full Title.
% \titlenote{with title note}             %% \titlenote is optional;
                                        %% can be repeated if necessary;
                                        %% contents suppressed with 'anonymous'
% \subtitle{Subtitle}                     %% \subtitle is optional
% \subtitlenote{with subtitle note}       %% \subtitlenote is optional;
                                        %% can be repeated if necessary;
                                        %% contents suppressed with 'anonymous'


%% Author with single affiliation.
\author{Thomas Sixuan Lou}
% \authornote{with author1 note}          %% \authornote is optional;
                                        %% can be repeated if necessary
% \orcid{nnnn-nnnn-nnnn-nnnn}             %% \orcid is optional
\affiliation{
  % \position{Position1}
  % \department{Department1}              %% \department is recommended
  \institution{University of Washington}            %% \institution is required
  % \streetaddress{Street1 Address1}
  % \city{City1}
  % \state{State1}
  % \postcode{Post-Code1}
  % \country{Country1}
}
\email{lous@cs.washington.edu}          %% \email is recommended

%% Paper note
%% The \thanks command may be used to create a "paper note" ---
%% similar to a title note or an author note, but not explicitly
%% associated with a particular element.  It will appear immediately
%% above the permission/copyright statement.
% \thanks{with paper note}                %% \thanks is optional
                                        %% can be repeated if necesary
                                        %% contents suppressed with 'anonymous'


%% Abstract
%% Note: \begin{abstract}...\end{abstract} environment must come
%% before \maketitle command
\begin{abstract}
% Text of abstract \ldots.
  This paper is an extension of the work by Apt \cite{Apt2000} to a pure object calculus.
\end{abstract}


%% 2012 ACM Computing Classification System (CSS) concepts
%% Generate at 'http://dl.acm.org/ccs/ccs.cfm'.
% \begin{CCSXML}
% <ccs2012>
% <concept>
% <concept_id>10011007.10011006.10011008</concept_id>
% <concept_desc>Software and its engineering~General programming languages</concept_desc>
% <concept_significance>500</concept_significance>
% </concept>
% <concept>
% <concept_id>10003456.10003457.10003521.10003525</concept_id>
% <concept_desc>Social and professional topics~History of programming languages</concept_desc>
% <concept_significance>300</concept_significance>
% </concept>
% </ccs2012>
% \end{CCSXML}

% \ccsdesc[500]{Software and its engineering~General programming languages}
% \ccsdesc[300]{Social and professional topics~History of programming languages}
%% End of generated code


%% Keywords
%% comma separated list
\keywords{Semantics, Constraint Logic Programming, Object Calculus}  %% \keywords is optional


%% \maketitle
%% Note: \maketitle command must come after title commands, author
%% commands, abstract environment, Computing Classification System
%% environment and commands, and keywords command.
\maketitle


\section{Introduction}

% Text of paper \ldots
At its core, constraint logic programming may be seen as a formalization of both
logic and computation by solving systems of relations. From a proof-theoretic
perspective, given a language $\mathcal{L}$, a logic on $\mathcal{L}$ is a
structure generated from a set of primitive rules, which may be seen as a set of
constraints on the language; while a constraint solver corresponds to the
computational interpretation of the logic. It's thus natural of using
constraints and substitutions (i.e. solutions to constraints) to give a
denotational semantics for the logic system we want to formalize.\\

This project will be an extension to the work by Apt \cite{Apt2000}, where he
gave a denotational semantics for first-order logic by embedding first-order
terms into constraints. He also proved the desired notion of ``truth'' in the
logic coincides with constraint resolution in this denotation. In this project,
we propose to incorporate objects into the logic so that we can solve local
constraints for each object - for example, an (insntance) solution to a
object is an (possibly non-ground) substitution of the object's local fields
satisfying constraints declared in the object itself.\\

The idea is to embed first-order terms with equality into each object and use
substitution for field update and constraint resolution. We will explore the usual object
oriented concepts \cite{Abadi1996} such as (possibly delegation and)
inheritance, as well as whether it is possible to establish the soundness of
such denotation.\\

Team member will include myself.

\section{A Pure Calculus with Constraints}

In this section, we will define $\textbf{O}_c$, an pure object-based calculus with
constraints introduced through a special binder $\iota(o).e$.\\

The calculus is parametrized over an algebra and an theory on that algebra
(definitions below are similar to those formulated in \cite{burris1981course, Birkhoff_1935}).
Intuitively, the algebra describes what are the primitive expressions in our
calculus. An theory defines a equivalence relation on that algebra, which our
calculus should be invariant under. In addition, the solver will solve
constraints with respect to the theory defined on the algebra.

\subsection{Algebra and Theory}

\begin{defn}[Type]
  A set of function symbols $\mathscr{F}$ is called an \emph{type} if for every
  $f \in \mathscr{F}$, it is assigned a non-negative integer $n$, called the
  \emph{arity} of $f$. Let $\mathscr{F}_k$ denote the set of function symbols
  with arity $k$.
\end{defn}

\begin{defn}[Algebra]
  An \emph{algebra} $\mathcal{A}$ of type $\mathscr{F}$ is a pair $\langle A , F
  \rangle$, where $A \neq \varnothing$ is called a \emph{carrier set} and $F$ is
  a set of finitary operations on $A$ so that for every $n$-ary function symbol
  $f \in \mathscr{F}_{n}$, there is an $n$-ary operation $f^{\mathcal{A}}$ on $A$
  (i.e. $f^{\mathcal{A}} : A^{n} \to A$).

  The map from every $f \in \mathscr{F}$ to $f^{\mathcal{A}}$ is called an \emph{interpretation}.

  The superscript $(-)^{\mathcal{A}}$ is often omitted when the context is
  clear.
\end{defn}

\begin{defn}[Language]
  Given a set of set of distinct objects $\textbf{X}$ (variables), and a type
  $\mathscr{F}$, a language $\mathcal{L}(\textbf{X},\mathscr{F})$ of type
  $\mathscr{F}$ over $\textbf{X}$ is the smallest set such that
  \begin{enumerate}
    \item $\textbf{X} \cup \mathscr{F}_0 \subset \mathcal{L}(\textbf{X},\mathscr{F})$
    \item if $p_1 , \ldots , p_n \in \mathcal{L}(\textbf{X},\mathscr{F})$, $f \in \mathscr{F}_n$, then the
      string $f(p_1, \ldots, p_n) \in \mathcal{L}(\textbf{X},\mathscr{F})$.
  \end{enumerate}

  The parameters $\textbf{X}$ and $\mathscr{F}$ are omitted when the context is clear.
\end{defn}

\begin{defn}[Theory]
  Given type $\mathscr{F}$, a \emph{theory} ``$\sim$'' is a relation between terms
  in $\mathcal{L}(\textbf{X},\mathscr{A})$, written as $p(x_1, \ldots, x_n) \sim
  q(x_1, \ldots, x_n)$. Arguments are often omitted when we are not making
  pointwise statements.\\

  An algebra $\mathcal{A}$ of type $\mathscr{F}$ satisfies $p \sim q$ if and
  only if
  \[
    \forall a_1 , \ldots a_n \in A \left[ p^{\mathcal{A}}(a_1 , \ldots , a_n) =
      q^{\mathcal{A}}(a_1 , \ldots , a_n) \right]
  \]
\end{defn}

\begin{exa}[Modular arithmetic]\label{E:modz}
  The arithmetic algebra $\mathbb{Z}/_{m} = \langle \mathbb{Z} , 0, 1, m, +, \times \rangle$ is
  an algebra over with carriers of integers $\mathbb{Z}$, characteristic
  elements $0,1 : \varnothing \to \mathbb{N}$ and the following binary operations:
  \[
    +,  \times      : \mathbb{Z}^2 \to \mathbb{Z}
  \]
  satisfying
  \begin{alignat*}{3}
      x + y & \sim x \times y & x \times y & \sim y \times x\\
      0 + x & \sim x          & x + (y + z) & \sim (x + y) + z\\
      1 \times x & \sim x     & x \times (y \times z) & \sim (x \times y) \times z\\
      x \times (y + z) & \sim x \times y + x \times z \quad & x + y \times m & \sim x
  \end{alignat*}
\end{exa}

\begin{exa}[Boolean algebra]
  A distributive lattice $\mathcal{L}$ on set $L$ is the algebra $\langle L,
  \land, \vee \rangle$ with operations $\land, \vee : L^2 \to L$ satisfying
  \begin{gather*}
    \begin{aligned}
      x \land x & \sim x\\
      x \land y & \sim y \land x\\
      x \vee y  & \sim y \vee x\\
      x \land (y \land z) & \sim (x \land y) \land z\\
      x \vee (y \vee z) & \sim (x \vee y) \vee z
    \end{aligned}
    \qquad
    \begin{aligned}
      x \vee x & \sim x\\
      x \vee (x \land y) & \sim x\\
      x \land (x \vee y) & \sim x\\
      x \land (y \vee z) & \sim (x \land y) \vee (x \land z)\\
      x \vee (y \land z) & \sim (x \vee y) \land (x \vee z)
    \end{aligned}
  \end{gather*}

  A boolean algebra $\mathcal{B} = \langle B, 0, 1, \land, \vee, ' \rangle$ is an
  extension to $\mathcal{L}$ with characteristic elements $0,1$ and an complement
  operator $' : B \to B$, satisfying
  \begin{equation*}
    \begin{aligned}
      x \land 0 \sim 0\\
      x \vee 1 \sim 1\\
      x' \land x \sim 0
    \end{aligned}
    \qquad
    \begin{aligned}
      x \land 1 \sim x\\
      x \vee 0 \sim x\\
      x' \vee x \sim 1
    \end{aligned}
  \end{equation*}

  An instance of boolean algebra is $\textbf{Su}(X)$ (subsets of $X$) with $B =
  \mathcal{P}(X)$, $0 = \varnothing$, $1$ is any singleton set. The theory on
  $\textbf{Su}(X)$ is interpreted as (note this of the semantics portion, which
  is independent from the syntactic/algebraic definitions):
  \[
    x \sim y \text{ iff } \forall z \left[ z \in x \leftrightarrow z \in y \right]
  \]
\end{exa}

In this paper, the constraint solver is abstracted as a blackbox that is able to
find a single best solution from a constraint whenever possible. The solver is
also parametrized over an algebra and a theory so that it's general enough to
solve equivalence constraints of arbitrary interpretations.\\

% \begin{defn}[Constraint]
  
% \end{defn}

For example, an solver $\mathfrak{S}$ over $\mathbb{Z}/_{7}$ is expected to
be able to obtain a solution from constraints on the relation $\pmod{7}$: given
constraint $c = x \sim 0 \land y \sim x$ (where $x$ and $y$ are objects in the
set of variable $\textbf{X}$), it should return a solution $\{ x \mapsto 7 , y
\mapsto 0\}$. Which solution to return is up to specific implementations of
solvers.

\subsection{Syntax Formalization}

We now formally introduce the calculus. The syntax of the calculus is
constructed from a language $\mathcal{L}(\textbf{X},\mathscr{F})$, and various special syntactical
constructs. There are two syntactical categories, the object fragment
(``$a$'') and the constraint fragment (``$c$'',``$e$''). They are separated so that rules
involving constraint solving are easier to write.\\

The complete syntax definition is given in \Cref{Oc:syntax}. Each object is
declared as a collection of $\langle \text{field name} , \text{method} \rangle$
and $\langle \text{constraint name} , \text{constraint} \rangle$ pairs. Each
method (resp. constraint) is given as a binder ``$\varsigma(x).a$'' (resp.
``$\iota(x).e$''), where $x$ is bound to which object it's invoked from.\\

Constraints in our calculus are of the form $\iota(x) . e$, where $x$ is a
variable which will bind to the object this constraint is invoked from before
solving the constraint term $e$. Constraint terms are constructed from atomic
$\sim$-relations and using connectives including $\land$, $\vee$ and $\neg$.
Notice, there are no $\exists$ quantifications on constraint terms since every
name(label) is already existentially quantified in each of the object.\\

We define free variables of $\textbf{O}_c$ terms and substitutions in
\Cref{Oc:fv} and \Cref{Oc:subst} respectively.\\

\begin{figure*}[t]
% \begin{mdframed}
  \centering
  \begin{alignat*}{3}
    \begin{aligned}
      c   & ::= \iota(x) . e && \text{constraint binder}\\
      e   & ::= && \text{constraint terms}\\
          & a_1 \sim a_2   && \quad \text{relation}\\
          & e_1 \land e_2  && \quad \text{conjunction}\\
          & e_1 \vee e_2   && \quad \text{disjunction}\\
          & \neg e_1       && \quad \text{negation}\\
          % & \exists l . e_1 && \quad \text{decl of local variable}\\
     \mathcal{E}[-] & && \text{context}
    \end{aligned}
    \quad
    \begin{aligned}
      a,b & ::= && \text{terms}\\
          & t(a_1, \ldots, a_n)   && \quad \text{primitive terms } t \in \mathcal{L} \\
          % & l,r  && \quad \text{labels}\\
          & x    && \quad \text{variable}\\
          & a.l_i \quad a.r_i && \quad \text{method/constraint invocation}\\
          & [ l_i = \varsigma (x) . b_i , r_j = c_j ]_{\substack{i = 1..n\\j = 1..m}} && \quad \text{object}\\
          & a.l_i \Leftarrow \varsigma (x) . b \quad a.r_i \Leftarrow c && \quad \text{method/constraint update}
    \end{aligned}
  \end{alignat*}
  \caption{Syntax of $\textbf{O}_c$ under theory $\mathcal{T}$ on algebra $\mathcal{A}$}
  \label{Oc:syntax}
% \end{mdframed}
\end{figure*}

\begin{figure*}[h]
  \centering
  \begin{alignat*}{2}
    \begin{aligned}
      FV(t(a_1 , \ldots , a_n)) & = \bigcup_{i \in [n]} FV(a_i)\\
      FV(x) & = \{x\}\\
      FV(l) & = \{l\}\\
      FV(a.l_i) & = FV(a) \cup \{l_i\}\\
      FV(a.r_i) & = FV(a) \cup \{r_i\}\\
      FV([l_i = \varsigma(x) . b_i , r_j = c_j]_{\substack{i \in [n]\\j \in [m]}})
      & = \bigcup_{i \in [n]} FV(\varsigma(x) . b_i) \cup \bigcup_{j \in [m]} FV(c_j)\\
      FV(a.l_i \Leftarrow \varsigma(x) . b) & = FV(a) \cup FV(\varsigma(x) . b)
    \end{aligned}
    \quad
    \begin{aligned}
      FV(a.r_i \Leftarrow c) & = FV(a) \cup FV(c)\\
      FV(\varsigma(x) . b) & = FV(b) \backslash \{x\}\\
      FV(\iota (x) . e) & = FV(e) \backslash \{x\}\\
      FV(a_1 \sim a_2) & = FV(a_1) \cup FV(a_2)\\
      FV(e_1 \land e_2) & = FV (e_1) \cup FV(e_2)\\
      FV(e_1 \vee e_2) & = FV (e_1) \cup FV(e_2)\\
      FV(\neg e) & = FV (e)
      % FV(\exists l . e) & = FV(e) \backslash \{l\}
    \end{aligned}
  \end{alignat*}
  \caption{Free variables in $\textbf{O}_c$}
  \label{Oc:fv}
\end{figure*}

\begin{figure*}[h]
  \centering
  \begin{alignat*}{2}
    \begin{aligned}
      & [a/y]t(a_1 , \ldots , a_n)  = t([a/y]a_1 , \ldots , [a/y]a_n)\\
      & [a/y]x  = x\\
      & [a/y]y  = a\\
      & [a/y]b.l_i  = ([a/y]b).l_i\\
      & [a/y]b.r_i  = ([a/y]b).r_i\\
      & [a/y]\varsigma(x) . b  = \varsigma(z) [a/y][z/x]b\\
      & [a/y]\iota(x) . b  = \iota(z) [a/y][z/x]b\\
      & [a/y][l_i = \varsigma(x) . b_i , r_j = c_j]  = [l_i = [a/y]\varsigma(x) . b_i , r_j = [a/y]c_j]
    \end{aligned}
    \quad
    \begin{aligned}
      & [a_1/y]a.l_i \Leftarrow \varsigma(x) . b  = ([a_1/y]a).l_i \Leftarrow [a_1/x]\varsigma(x). b\\
      & [a_1/y]a.r_i \Leftarrow c  = ([a_1/y]a).r_i \Leftarrow [a_1/x]c\\
      & [a/y]a_1 \sim a_2  = [a/y]a_1 \sim [a/y]a_2\\
      & [a/y]e_1 \land e_2  = [a/y]e_1 \land [a/y]e_2\\
      & [a/y]e_1 \vee e_2  = [a/y]e_1 \vee [a/y]e_2\\
      & [a/y]\neg e  = \neg ([a/y]e)
      % [a/y]\exists l . e & = \exists z . ([a/y][z/l]e)\\
    \end{aligned}
  \end{alignat*}
  \caption{Substitution in $\textbf{O}_c$}
  \label{Oc:subst}
\end{figure*}


Since our calculus was parametrized over arbitrary algebras and theories, we
define the notion of evaluation that evaluate a term involving primitives (i.e.
with terms in $\mathcal{L}$) down to its general form by substituting primitives
with terms and formulas in the algebra.\\

In evaluating terms using the equational facts from the algebra, we have avoided
adding rules of evaluating arbitrary terms into our calculus. Thus, we only need
to focus the core object and constraint solving fragment in our calculus.

\begin{defn}[Evaluation]
  Evaluation is the extended notion of interpretation from languages to algebras.

  Let $\mathcal{L}(\textbf{X}, \mathscr{F})$ be a language, let $\mathcal{A} = \langle A , F \rangle$ be
  an algebra. A \emph{evaluation} of term $a$ is defined inductively in
  \Cref{Oc:evaluation}. The evaluated term is called a \emph{generalized term}.

  \begin{figure*}[h]
    \centering
    \begin{alignat*}{2}
      \begin{aligned}
        & \eval{f(a_1 , \ldots, a_2)} = f^{\mathcal{A}}(\eval{a_1} , \ldots , \eval{a_n})\\
        & \eval{x} = x\\
        & \eval{a.l_i} = \eval{a}.l_i\\
        & \eval{a.r_i} = \eval{a}.r_i\\
        & \eval{[l_i = \varsigma(x) . b_i , r_i = c_i]} =  [l_i = \varsigma(x) . \eval{b_i} , r_i = \eval{c_i}]\\
        & \eval{a.l_i \Leftarrow \varsigma (x) . b} = \eval{a}.l_i \Leftarrow \varsigma(x) . \eval{b}\\
        & \eval{a.r_i \Leftarrow c} = \eval{a}.l_i \Leftarrow \eval{c}
      \end{aligned}
          \quad
      \begin{aligned}
        & \eval{\iota(x) . e} = \iota(x) . \eval{e}\\
        & \eval{a_1 \sim a_2} = \iota(a_1) \sim \iota(a_2)\\
        & \eval{e_1 \land e_2} = \eval{e_1} \land \eval{e_2}\\
        & \eval{e_1 \vee e_2}   = \eval{e_1} \vee \eval{e_2}\\
        & \eval{\neg e}         = \neg \eval{e}\\
        % & \eval{\exists l . e } = \exists l . \eval{e}\\
      \end{aligned}
    \end{alignat*}
    \caption{Evaluation}
    \label{Oc:evaluation}
  \end{figure*}

\end{defn}

\subsection{Semantics of Constraints}

Let $\solver$ be a solver parametrized by an algebra $\mathcal{A}$ and a theory
$\mathcal{T}$. It interacts with our calculus by the following satisfaction
judgment $o(c) \models_{\solver} S$, where $o$ is an object, $S$ is a set of
objects and $c$ is a constraint. The judgment reads: $S$ is the set of objects that
satisfies constraint $c$ when evaluated from an particular object $o$.\\

This satisfaction judgment gives the semantics of equations (more generally,
$\sim$-relations) in our calculus. If we leave out the constraint $c$ and the
initial object $o$ is the judgment, it induces a semantic map on constraints:
\[
  \llbracket c \rrbracket : Object \to \mathcal{P}(Object)
\]
Therefore, having defined the satisfaction judgment of constraints on objects,
the semantics for constraints are defined through the following relationship:
\[
  o(c) \models_{\solver} \llbracket c \rrbracket (o)
\]
On the other hand, the judgment is an externalization of the semantics of
constraints, this formulation hides away the computational contents in interpreting
constraints.\\

We may sometimes want to construct a new construct $c$ by combining several
constraints $\{c_i\}_{i \in I}$. Aside from creating a new object with all of
$c_i$ inserted, we may also internalize it using first-order connectives we've
defined earlier.  This leads us to define the following auxiliary
functions:
\begin{align*}
  (\iota(x) . e_1) \otimes (\iota(x) . e_2) \triangleq \iota(x) . e_1 \land e_2\\
  (\iota(x) . e_1) \oplus (\iota(x) . e_2) \triangleq \iota(x) . e_1 \vee e_2
\end{align*}
There is another series of design decision to make involving when the solver is
fired, when the object is updated and how the object's methods are updated in
solving constraints. In our formulation of the calculus, we insist the
following:

\begin{enumerate}
  \item The only place the solver interacts with the satisfaction judgment is on
    the resolution of $\sim$-relations that present as premises of multiple satisfaction rules.
  \item Constraints are solved eagerly, meaning (1) on constraint invocation the
    rule $\textsc{red-c-update}$ is fired (given in the next section) ensuring
    the resulting object satisfies the newly declared constraint and (2) on
    method updates, the rule $\textsc{red-m-update}$ is fired to get a new
    solution from the newly updated method.
  \item Objects are updated monotonically, on constraint update, we do not
    attempt to backtrack and first solve with the rest of the constraints,
    instead we simply run the solver on the current object (solution).
  \item Method updates are all handled functionally, a new object is returned on
    each invocation to the satisfaction judgment. Thus, there is no notion of
    references or object identities in this calculus. And we don't need to keep
    track the object identities environment.
\end{enumerate}

\begin{figure*}[t]
  \centering
  \begin{spreadlines}{10pt}
  \begin{gather*}
    \infer[\textsc{sat-eq}]
    {o(\iota(x) . a_1 \sim a_2) \models_{\solver} \{o\}}
    {   [o/x]a_1 \twoheadrightarrow v_1 \in \mathcal{A}
      & [o/x]a_2 \twoheadrightarrow v_2 \in \mathcal{A}
      & v_1 \sim v_2}\\
    \infer[\textsc{sat-empty}]
    {o(\iota(x) . a_1 \sim a_2) \models_{\solver} \varnothing}
    {   [o/x]a_1 \in \mathcal{A}
      & [o/x]a_2 \in \mathcal{A}
      & [o/x]a_1 \twoheadrightarrow v_1
      & [o/x]a_2 \twoheadrightarrow v_2
      & v_1 \not\sim v_2}\\
    \infer[\textsc{sat-update}_1]
    {o(\iota(x) . a_1 \sim a_2) \models_{\solver} \{o.l \Leftarrow \varsigma(z) . v_1\}}
    {\deduce
      {o.l = [o/x]a_1}
      {l \not\in FV([o/x]a_2)}
      &
      \deduce
      {l \text{ is a label in } o}
      {[o/x]a_2 \to v_2}
      & v_2 \sim v_1}\\
    \infer[\textsc{sat-update}_2]
    {o(\iota(x) . a_1 \sim a_2) \models_{\solver} \{o.l \Leftarrow \varsigma(z) . v_2\}}
    {\deduce
      {v_1 \text{ is not a label or a variable}}
      {l \text{ is a label in } o}
      &
      \deduce
      {o.l = [o/x]a_2}
      {l \not\in FV(\hat{a}_1)}
      &
      \deduce
      {\hat{a}_1 \to v_1}
      {v_1 \sim v_2}}\\
  \end{gather*}
  \end{spreadlines}
  \caption{Satisfaction Judgment}
  \label{Oc:sat}
\end{figure*}

\begin{figure*}[t]
  \centering
  \begin{spreadlines}{10pt}
  \begin{gather*}
    \infer[\textsc{sat-conj}]
    {o(\iota(x) . e_1 \land e_2) \models_{\solver} S}
    {o(\iota(x) . e_1) \models_{\solver} S_1
      & S_1(\iota(x) . e_2) \models_{\solver} S}\\
    \infer[\textsc{sat-disj}]
    {o(\iota(x) . e_1 \vee e_2) \models_{\solver} S_1 \cup S_2}
    {o(\iota(x) . e_1) \models_{\solver} S_1
      & o(\iota(x) . e_2) \models_{\solver} S_2}\\
    \infer[\textsc{sat-neg}_1]
    {o(\iota(x) . \neg e) \models_{\solver} \varnothing}
    {o(\iota(x) . e) \models_{\solver} S
      & o \in S}\\
    \infer[\textsc{sat-neg}_2]
    {o(\iota(x) . \neg e) \models_{\solver}  \{o\} }
    {o(\iota(x) . \neg e) \models_{\solver} \varnothing}
  \end{gather*}
  \end{spreadlines}
  \caption{Satisfaction Judgment on First-order Constraints}
  \label{Oc:sat-foc}
\end{figure*}

\subsection{Operational Semantics: Reduction}

Now, we are ready to define the notion of reduction given the semantics of
constraints. $o \rightarrowtail o'$ is a single step reduction between two
top-level terms, $o \to o'$ is a single step reduction between two
terms within a context and $\twoheadrightarrow$ is the reflexive, transitive
closure of $\to$. The reduction judgment serves as the operational semantics of
$\textbf{O}_c$.

\begin{figure*}[t]
  \centering
  \begin{spreadlines}{10pt}
  \begin{gather*}
    \infer[\textsc{red-m-invoke}]{[l_i = \varsigma(x) . b_i , r_j = c_j].l_k
      \rightarrowtail [[l_i = \varsigma(x) . b_i , r_i = c_i]_{\substack{i \in [n]\\j \in [m]}} / x] b_k}{}\\
    \infer[\textsc{red-c-invoke}]
    {[l_i = \varsigma(x) . b_i , r_j = c_j].c_k \rightarrowtail o'}
    {[l_i = \varsigma(x) . b_i , r_j = c_j](c_k) \models_{\solver} S
      & o' \in S}\\
    \infer[\textsc{red-m-update}]
    {[l_i = \varsigma(x) . b_i , r_j = c_j].l_k \Leftarrow \varsigma(x) . b
      \rightarrowtail o'}
    {[l_k = \varsigma(x) . b , l_{i \neq k} = \varsigma(x) . b_i , r_j =
      c_j](\bigotimes_{i \in [m]} c_i ) \models_{\solver} S
      & o' \in S}\\
    \infer[\textsc{red-c-update}]
    {[l_i = \varsigma(x) . b_i , r_j = c_j].c_k \Leftarrow c \rightarrowtail o'}
    {[l_i = \varsigma(x) . b_i , r_{j \neq k} = c_j , r_k = c](c) \models_{\solver} S
      & o' \in S}\\
    \infer[\textsc{red-congr}]
    {\ctxt{a} \to \ctxt{a'}}
    {a \rightarrowtail a'}
  \end{gather*}
  \end{spreadlines}
  \caption{Reduction relation in $\textbf{O}_c$}
  \label{Oc:reduction}
\end{figure*}

Now having defined both the satisfaction judgment and the reduction relation, we
present several examples of derivations in the calculus to provide some
intuitions.

\begin{exa}
  Suppose we are working over algebra $\mathbb{N}/_{7}$ - modular arithmetic over
  natural numbers (similar to $\mathbb{Z}/_{m}$ as we introduced in
  \ref{E:modz} but on $\mathbb{N}$ instead of $\mathbb{Z}$). Suppose we have an
  object $pt = [px = \varsigma(x) . (x . px) , py = \varsigma(x) . (x . py) , c
  = \iota(x) . (x.px \sim x.py)]$, which represents a point whose $x$ and $y$
  coordinates are congruent modulo $7$. We want to show the following reduction
  is derivable in our calculus:
  \[
    pt.px \Leftarrow \varsigma(x) . 2 \twoheadrightarrow pt'
  \]
  where $pt' = [px = \varsigma(x) . 2 , py = \varsigma(x) . 9, c = \iota(x) .
  (x.px \sim x.py)]$.
\end{exa}

\begin{proof}
  Let $o = [px = \varsigma(x) . 2 , py = \varsigma(x) . (x.py), c = \iota(x) .
  (x.px \sim x.py)]$. We first derive $pt.px \Leftarrow(x) . 2 \to o.py
  \Leftarrow \varsigma(x) . 9$, then derive $o.py \Leftarrow \varsigma(x). 9 \to
  pt'$, then by transitivity, we have our result.
  \[
    \infer[\textsc{red-m-update}]
    {pt.px \Leftarrow \varsigma(x) . 2 \to o.py \Leftarrow \varsigma(x) . 9}
    {\infer[\textsc{sat-update}_2]
      {o(\iota(x) . (x.px \sim x.py)) \models_{\solverP{\mathbb{N}/_{7}}} \{o.py
        \Leftarrow \varsigma(x) . 9\}}
      {
        \deduce
        {
          \deduce
          {
            \deduce
            {
              \deduce
              {
                [o/x](x.py) = o.py
              }
              {
                [o/x](x.px) = o.px
              }
            }
            {
              py \notin FV(o.px)
            }
          }
          {
            py \text{ is a label in } o
          }
        }
        {
          2 \sim 9
        }
        &
        \infer[\textsc{red-m-invoke}]
        {o.px \rightarrowtail 2} {}
      }
    }
  \]
  \[
    \infer[\textsc{red-m-update}]
    {o.py \Leftarrow \varsigma(x) . 9 \to pt'}
    {
      \infer[\textsc{sat-eq}]
      {
        pt'(\iota(x) . (x.px \sim x.py)) \models \{pt'\}
      }
      {
        \deduce
        {[pt'/x](x.px) = pt'.px}
        {[pt'/x](x.py) = pt'.py}
        &
        \deduce
        {\infer{pt'.px \rightarrowtail 2}{}}
        {\infer{pt'.px \rightarrowtail 9}{}}
        &
        2 \sim 9
      }
    }
  \]
\end{proof}

\begin{thm}[monotonicity of objects]
  For any object $o = [l_i = \varsigma(x) . b_i , r_j = c_j]_{i \in [n], j \in [m]}$, given any
  constraint $c$, indices $p,q \in [m]$, if $o(c_p) \models_{\solver} S$ and $o.r_q
  \Leftarrow c \rightarrowtail o'$, then $o' \in S$.
\end{thm}

\section{Typing the Pure Calculus}

In this section, we will type the untyped calculus $\textbf{O}_c$ as defined in the previous
section, and prove the reduction judgment preserves typing relation (subject
reduction). We will first type the core calculus (with familiar record typing),
and then extend the calculus with object-oriented features such as inheritance
and a formulation of dynamic constraint dispatch.

\subsection{Typing the Core Calculus}

The $\textbf{O}_{ct}$, the type system induced from the core calculus will be
really familiar to those who have implemented objects using records in
functional languages. This type system is similar to usual record semantics in
the sense that (1) we are not dealing with object identities or references (2)
state updates are all handled functionally (i.e. on every method update or
constraint satisfaction, a new object with desired fields are returned) and (3)
we have restricted ourselves to only solve $\sim$-constraints between primitive
terms in a given algebra (so that there are no higher-order unification
happening).\\

\begin{figure*}[t]
  \centering
  \begin{alignat*}{3}
    \begin{aligned}
      A,B & ::=                     && \text{types}\\
          & \mathcal{A}             && \quad \text{atomic types}\\
          & [l_i : B_i , r_j : C_j] && \quad \text{object type}\\
          & Id(\mathcal{A})         && \quad \text{Identity Type in } \mathcal{A}\\
    \end{aligned}
    \quad
    \begin{aligned}
      \Gamma & ::= && \text{Context}\\
             & \langle  \rangle && \quad \text{empty context}\\
             & \Gamma , x : A   && \quad \text{context augmentation}
    \end{aligned}
  \end{alignat*}
  \caption{Syntax of types $\textbf{O}_{ct}$ under theory $\mathcal{T}$ on algebra $\mathcal{A}$}
  \label{Oct:syntax}
\end{figure*}

In addition to the syntax of types presented in \ref{Oct:syntax}, we also need
to modify the syntax for terms from $\textbf{O}_c$ to $\textsc{O}_{ct}$ by
annotating types in methods. Therefore, what was of the form $\varsigma(x) . b$
before should be annotated into term of the form $\varsigma(x : A) . b$, where
$A$ is the type of the object $x$ binds to.

\begin{figure*}[h]
  \centering
  \begin{spreadlines}{10pt}
  \begin{gather*}
    \infer[\textsc{ctxt-empty}]
    {\ectxt \vdash}
    {}
    \quad
    \infer[\textsc{ctxt-aug}]
    {\Gamma , x : A \vdash}
    {\Gamma \vdash A}
    \quad
    \infer[\textsc{ty-atomic}]
    {\Gamma \vdash \mathcal{A}}
    {}\\
    \infer[\textsc{ty-id}]
    {\Gamma \vdash Id(\mathcal{A})}
    {}
    \quad
    \infer[\textsc{ty-obj}]
    {\Gamma \vdash [l_1 : B_1 , \ldots , l_n : B_n]}
    {\Gamma \vdash B_i  \qquad \forall i \in [n]}\\
    \infer[\textsc{id-refl}]
    {\Gamma \vdash a \sim b : Id(\mathcal{A})}
    {\Gamma \vdash a : \mathcal{A}
      & \Gamma \vdash b : \mathcal{A}}
    \quad
    \infer[\textsc{id-trans}]
    {\Gamma \vdash a_1 \sim a_3 : Id(\mathcal{A})}
    {\Gamma \vdash a_1 \sim a_2 : Id(\mathcal{A})
      & \Gamma \vdash a_2 \sim a_3 : Id(\mathcal{A})}
    \quad
    \infer[\textsc{id-symm}]
    {\Gamma \vdash a_2 \sim a_1 : Id(\mathcal{A})}
    {\Gamma \vdash a_1 \sim a_2 : Id(\mathcal{A})}\\
    % \infer[\textsc{id-atomic}]
    % {t_1(a_1 , \ldots , a_n) \sim t_2(b_1 , \ldots , b_n) : Id(\mathcal{A})}
    % { \Gamma \vdash t_1(a_1 , \ldots, a_n) : \mathcal{A}
    %   & \Gamma \vdash t_2(b_1 , \ldots , b_n) : \mathcal{A}
    % }
    % \quad
    % \infer[\textsc{id-var}_1]
    % {\Gamma \vdash o.l \sim t(a_1 , \ldots , a_n) : Id(\mathcal{A})}
    % {\Gamma \vdash o.l : \mathcal{A}
    %   & \Gamma \vdash t(a_1, \ldots, a_n) \sim t(a_1, \ldots, a_n) : Id(\mathcal{A}) }\\
    % \infer[\textsc{id-var}_2]
    % {\Gamma \vdash t(a_1 , \ldots , a_n) \sim o.l : Id(\mathcal{A})}
    % {\Gamma \vdash o.l : \mathcal{A}
      % & \Gamma \vdash t(a_1, \ldots, a_n) \sim t(a_1, \ldots, a_n) :
      % Id(\mathcal{A})}
    % \quad
    \infer[\textsc{id-neg}]
    {\Gamma \vdash \neg e : Id(\mathcal{A})}
    {\Gamma \vdash e : Id(\mathcal{A})}
    \quad
    \infer[\textsc{id-conj}]
    {\Gamma \vdash e_1 \land e_2 : Id(\mathcal{A})}
    {\Gamma \vdash e1 : Id(\mathcal{A}) & \Gamma \vdash e2 : Id(\mathcal{A})}
    \quad
    \infer[\textsc{id-disj}]
    {\Gamma \vdash e_1 \vee e_2 : Id(\mathcal{A})}
    {\Gamma \vdash e1 : Id(\mathcal{A}) & \Gamma \vdash e2 : Id(\mathcal{A})}\\
    \infer[\textsc{tm-var}]
    {\Gamma \vdash x : A}
    {x : A \in \Gamma}
    \quad
    \infer[\textsc{tm-atomic}]
    {\Gamma \vdash t(a_1, \ldots, a_n) : \mathcal{A}}
    {t \in \mathscr{F}
    & \Gamma \vdash a_i : \mathcal{A} \quad \forall i \in [n]}\\
    \infer[\textsc{tm-obj}]
    {\Gamma \vdash [l_i = \varsigma(x : A) . b_i , r_j = \iota(x : A) . e_j] : [l_i : B_i]}
    {\Gamma , x : A \vdash b_i : B_i \ \forall i \in [n]
      & \Gamma , x : A \vdash e_j : Id(\mathcal{A}) \ \forall j \in [m]
      & A = [l_i : B_i]}\\
    \infer[\textsc{tm-m-invoke}]
    {\Gamma \vdash [l_i = \varsigma(x : A) . b_i , r_j = c_j].l_k : B_k}
    {\Gamma \vdash [l_i = \varsigma(x: A) . b_i , r_j = c_j] : [l_i : B_i]
      & k \leq n}
    \quad
    \infer[\textsc{tm-c-invoke}]
    {\Gamma \vdash [l_i = \varsigma(x : A) . b_i , r_j = c_j] : Id(\mathcal{A})}
    {\Gamma \vdash [l_i = \varsigma(x : A) . b_i , r_j = c_j] : [l_i = B_i]}\\
    \infer[\textsc{tm-m-update}]
    {\Gamma \vdash [l_i = \varsigma(x : A) . b_i , r_j = c_j].l_k \Leftarrow
      \varsigma(x : A) . b : [l_i : B_i]}
    {\Gamma , x : A \vdash b : B_k
      & \Gamma \vdash [l_i = \varsigma(x : A) . b_i , r_j = c_j] : [l_i : B_i]}\\
    \infer[\textsc{tm-c-update}]
    {\Gamma \vdash [l_i = \varsigma(x : A) . b_i , r_j = c_j].r_k \Leftarrow c :
    [l_i : B_i]}
    {\Gamma \vdash [l_i = \varsigma(x : A) . b_i , r_j = c_j] : [l_i : B_i]}
  \end{gather*}
  \end{spreadlines}
  \caption{Typing Judgments in $\textbf{O}_{ct}$}
  \label{Oct:typing}
\end{figure*}

Few comments on typing rules...\\

\begin{lem}\label{L:bot}
  For arbitrary object $o = [l_i = \varsigma(x : A) . b_i , r_i = c_i]$, if
  field $l_k$ has method $\varsigma(x : A) . (x.l_k)$, then the field $l_k$ may
  be typed by any type $B$. This will be the analogous notion of $Falsum$ (i.e.
  bottom ``$\bot$'') in our calculus.
\end{lem}

\begin{proof}
  Without loss of generality, suppose our object only has one method field named
  $l$. Let $B$ be an arbitrary type, let $A = [l : B]$,
  \[
    \infer[\textsc{tm-obj}]
    {\Gamma \vdash [l = \varsigma(x : A) . (x.l)] : [l : B]}
    {\infer[\textsc{tm-m-invoke}]
      {\Gamma , x : [l : B] \vdash x.l : B}
      {\infer[\textsc{tm-var}]
        {\Gamma , x : [l : B] \vdash x : [l : B]}
        {}
      }
    }
  \]
\end{proof}


In order for our calculus to even make sense (in a sense that typing rules
interact well with other judgments), we need to prove several important
property. One of which is to show our constraints satisfaction judgment as we
defined in Section 2 preserves the typing judgment. Maybe we can call this
subject satisfaction (in reference to subject reduction).\\

Before we going into proof of the theorem, we define another syntactic apparatus, that is, the
typing judgment for a set of objects. Let $S$ be a set of objects, $A$ is a
type, we write $\Gamma \vdash S : A$ to mean for all $o \in S$, we have $\Gamma
\vdash o : A$. This theorem is proved below:

\begin{lem}[substitution preserves typing]\label{L:subst-typing}
  Suppose our calculus was parametrized over algebra $\mathcal{A}$ and theory
  $\mathcal{T}$. If $\Gamma , x : A \vdash t : T$ and $\ectxt \vdash v : A$,
  then $\Gamma \vdash [v/x]t : T$.
\end{lem}

\begin{proof}
  Induction on the structure of term $t$: we will only prove the case for
  primitives here, rest of the cases all follows directly by the definition of
  the substitution function and inductive hypotheses.
  \begin{enumerate}
    \item $t = t_1(a_1 , \ldots , a_n)$ for primitive term, since we have $\Gamma , x : A \vdash
      t_1(a_1, \ldots , a_n) : T$, it must be the case that $T = \mathcal{A}$
      and have premises:
      \[
        t \in \mathscr{F}
        \quad \text{and} \quad
        \Gamma , x : A \vdash a_i : \mathcal{A} \text{ for all } i \in [n]
      \]
      By IH, we have $\Gamma \vdash [v/x]a_i : \mathcal{A}$ for all $i \in [n]$.
      Since by the definition of substitution, we have
      \[
        [v/x](t_1(a_1 \ldots , a_n)) = t_1([v/x]a_1 , \ldots , [v/x]a_n)
      \]
      therefore we have
      \[
        \infer[\textsc{tm-atomic}]
        {\Gamma \vdash t_1([v/x]a_1 , \ldots , [v/x]a_n) : T}
        {t_1 \in \mathscr{F}
          & \Gamma \vdash [v/x]a_i : \mathcal{A} \quad \forall i \in [n]}
      \]
  \end{enumerate}
\end{proof}

\begin{thm}[subject satisfaction \& subject reduction]
  Suppose our calculus was parametrized over algebra $\mathcal{A}$ and theory
  $\mathcal{T}$.
  \begin{enumerate}
    \item If $o(\iota(x) . e) \models_{\solver} S$, $\Gamma \vdash e :
      Id(\mathcal{A})$ and $\Gamma \vdash o : A$, then $\Gamma \vdash S : A$.
    \item If $\Gamma \vdash e : A$ and $e \twoheadrightarrow e'$, then $\Gamma
      \vdash e' : A$.
  \end{enumerate}
\end{thm}

\begin{proof}
  Mutually induction on the derivation of $o(\iota(x) . e) \models_{\solver} S$
  and $\Gamma \vdash e : A$.
  \begin{enumerate}
    \item \textsc{sat-eq}, then $S = \{o\}$, $\Gamma \vdash o : A$ by assumption.
    \item \textsc{sat-empty}, then $S = \varnothing$, this case is vacuously
      true.
    \item $\textsc{sat-update}_1$, then $e = a_1 \sim a_2$, $S = \{o.l
      \Leftarrow \varsigma(z).v_1\}$, $[o/x]a_1 = o.l$, where $l$ is a label in
      $o$, since $o$ was well-typed, suppose the field $l$ has type $B$. Then it
      suffices to show $\Gamma , z : A \vdash v_1 : B$. Since the equation $a_1
      \sim a_2$ well typed with type $Id(\mathcal{A})$, we must have $\Gamma
      \vdash a_1 : \mathcal{A}$ and $\Gamma \vdash a_2 : \mathcal{A}$. By
      \Cref{L:subst-typing} we have both $\Gamma \vdash o.l : \mathcal{A}$ and
      $\Gamma \vdash [o/x]a_2 : \mathcal{A}$. By the subject reduction mutual
      inductive hypothesis, we have $\Gamma \vdash v_2 : \mathcal{A}$. By
      transitivity of $\sim$-relation, we have $v_1 : \mathcal{A}$ as wanted.
      % \begin{align*}
      %   v_1 \sim 
      % \end{align*}
      % with $o.l = [o/x]a_1$. Induction on
      % the judgment $\Gamma \vdash e : Id(\mathcal{A})$ together with the fact
      % that $o.l = [o/x]a_i$ gives us only one case to consider:
      % $\textsc{id-var}_1$. Then we have $\Gamma \vdash a_2 \in \mathcal{A}$, thus
      % $\Gamma \vdash [o/x]a_2 : \mathcal{A}$. Therefore, applying
      % \textsc{tm-m-update} gives us the conclusion $\Gamma \vdash o.l \Leftarrow
      % \varsigma(z) . [o/x]a_2 : A$.
  \end{enumerate}
\end{proof}

\subsection{Subtyping}

\subsection{Inheritance}

\subsection{Constraint Dispatch}

%% Bibliography
%\bibliography{bibfile}



%% Appendix
\appendix
\section{Appendix}

Text of appendix \ldots

\bibliographystyle{plain}
\bibliography{refs}

\end{document}