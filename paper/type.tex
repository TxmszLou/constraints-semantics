\section{Typing the Pure Calculus}

In this section, we will type the untyped calculus $\textbf{O}_c$ as defined in the previous
section, and prove the reduction judgment preserves typing relation (subject
reduction). We will first type the core calculus (with familiar record typing),
and then extend the calculus with object-oriented features such as inheritance
and a formulation of dynamic constraint dispatch.

\subsection{Typing the Core Calculus}

The $\textbf{O}_{ct}$, the type system induced from the core calculus will be
really familiar to those who have implemented objects using records in
functional languages. This type system is similar to usual record semantics in
the sense that (1) we are not dealing with object identities or references (2)
state updates are all handled functionally (i.e. on every method update or
constraint satisfaction, a new object with desired fields are returned) and (3)
we have restricted ourselves to only solve $\sim$-constraints between primitive
terms in a given algebra (so that there are no higher-order unification
happening).\\

\begin{figure*}[t]
  \centering
  \begin{alignat*}{3}
    \begin{aligned}
      A,B & ::=                     && \text{types}\\
          & \mathcal{A}             && \quad \text{atomic types}\\
          & [l_i : B_i , r_j : C_j] && \quad \text{object type}\\
          & Id(\mathcal{A})         && \quad \text{Identity Type in } \mathcal{A}\\
    \end{aligned}
    \quad
    \begin{aligned}
      \Gamma & ::= && \text{Context}\\
             & \langle  \rangle && \quad \text{empty context}\\
             & \Gamma , x : A   && \quad \text{context augmentation}
    \end{aligned}
  \end{alignat*}
  \caption{Syntax of types $\textbf{O}_{ct}$ under theory $\mathcal{T}$ on algebra $\mathcal{A}$}
  \label{Oct:syntax}
\end{figure*}

In addition to the syntax of types presented in \ref{Oct:syntax}, we also need
to modify the syntax for terms from $\textbf{O}_c$ to $\textsc{O}_{ct}$ by
annotating types in methods. Therefore, what was of the form $\varsigma(x) . b$
before should be annotated into term of the form $\varsigma(x : A) . b$, where
$A$ is the type of the object $x$ binds to.

\begin{figure*}[h]
  \centering
  \begin{spreadlines}{10pt}
  \begin{gather*}
    \infer[\textsc{ctxt-empty}]
    {\ectxt \vdash}
    {}
    \quad
    \infer[\textsc{ctxt-aug}]
    {\Gamma , x : A \vdash}
    {\Gamma \vdash A}
    \quad
    \infer[\textsc{ty-atomic}]
    {\Gamma \vdash \mathcal{A}}
    {}\\
    \infer[\textsc{ty-id}]
    {\Gamma \vdash Id(\mathcal{A})}
    {}
    \quad
    \infer[\textsc{ty-obj}]
    {\Gamma \vdash [l_1 : B_1 , \ldots , l_n : B_n]}
    {\Gamma \vdash B_i  \qquad \forall i \in [n]}\\
    \infer[\textsc{id-refl}]
    {\Gamma \vdash a \sim b : Id(\mathcal{A})}
    {\Gamma \vdash a : \mathcal{A}
      & \Gamma \vdash b : \mathcal{A}}
    \quad
    \infer[\textsc{id-trans}]
    {\Gamma \vdash a_1 \sim a_3 : Id(\mathcal{A})}
    {\Gamma \vdash a_1 \sim a_2 : Id(\mathcal{A})
      & \Gamma \vdash a_2 \sim a_3 : Id(\mathcal{A})}
    \quad
    \infer[\textsc{id-symm}]
    {\Gamma \vdash a_2 \sim a_1 : Id(\mathcal{A})}
    {\Gamma \vdash a_1 \sim a_2 : Id(\mathcal{A})}\\
    % \infer[\textsc{id-atomic}]
    % {t_1(a_1 , \ldots , a_n) \sim t_2(b_1 , \ldots , b_n) : Id(\mathcal{A})}
    % { \Gamma \vdash t_1(a_1 , \ldots, a_n) : \mathcal{A}
    %   & \Gamma \vdash t_2(b_1 , \ldots , b_n) : \mathcal{A}
    % }
    % \quad
    % \infer[\textsc{id-var}_1]
    % {\Gamma \vdash o.l \sim t(a_1 , \ldots , a_n) : Id(\mathcal{A})}
    % {\Gamma \vdash o.l : \mathcal{A}
    %   & \Gamma \vdash t(a_1, \ldots, a_n) \sim t(a_1, \ldots, a_n) : Id(\mathcal{A}) }\\
    % \infer[\textsc{id-var}_2]
    % {\Gamma \vdash t(a_1 , \ldots , a_n) \sim o.l : Id(\mathcal{A})}
    % {\Gamma \vdash o.l : \mathcal{A}
      % & \Gamma \vdash t(a_1, \ldots, a_n) \sim t(a_1, \ldots, a_n) :
      % Id(\mathcal{A})}
    % \quad
    \infer[\textsc{id-neg}]
    {\Gamma \vdash \neg e : Id(\mathcal{A})}
    {\Gamma \vdash e : Id(\mathcal{A})}
    \quad
    \infer[\textsc{id-conj}]
    {\Gamma \vdash e_1 \land e_2 : Id(\mathcal{A})}
    {\Gamma \vdash e1 : Id(\mathcal{A}) & \Gamma \vdash e2 : Id(\mathcal{A})}
    \quad
    \infer[\textsc{id-disj}]
    {\Gamma \vdash e_1 \vee e_2 : Id(\mathcal{A})}
    {\Gamma \vdash e1 : Id(\mathcal{A}) & \Gamma \vdash e2 : Id(\mathcal{A})}\\
    \infer[\textsc{tm-var}]
    {\Gamma \vdash x : A}
    {x : A \in \Gamma}
    \quad
    \infer[\textsc{tm-atomic}]
    {\Gamma \vdash t(a_1, \ldots, a_n) : \mathcal{A}}
    {t \in \mathscr{F}
    & \Gamma \vdash a_i : \mathcal{A} \quad \forall i \in [n]}\\
    \infer[\textsc{tm-obj}]
    {\Gamma \vdash [l_i = \varsigma(x : A) . b_i , r_j = \iota(x : A) . e_j] : [l_i : B_i]}
    {\Gamma , x : A \vdash b_i : B_i \ \forall i \in [n]
      & \Gamma , x : A \vdash e_j : Id(\mathcal{A}) \ \forall j \in [m]
      & A = [l_i : B_i]}\\
    \infer[\textsc{tm-m-invoke}]
    {\Gamma \vdash [l_i = \varsigma(x : A) . b_i , r_j = c_j].l_k : B_k}
    {\Gamma \vdash [l_i = \varsigma(x: A) . b_i , r_j = c_j] : [l_i : B_i]
      & k \leq n}
    \quad
    \infer[\textsc{tm-c-invoke}]
    {\Gamma \vdash [l_i = \varsigma(x : A) . b_i , r_j = c_j] : Id(\mathcal{A})}
    {\Gamma \vdash [l_i = \varsigma(x : A) . b_i , r_j = c_j] : [l_i = B_i]}\\
    \infer[\textsc{tm-m-update}]
    {\Gamma \vdash [l_i = \varsigma(x : A) . b_i , r_j = c_j].l_k \Leftarrow
      \varsigma(x : A) . b : [l_i : B_i]}
    {\Gamma , x : A \vdash b : B_k
      & \Gamma \vdash [l_i = \varsigma(x : A) . b_i , r_j = c_j] : [l_i : B_i]}\\
    \infer[\textsc{tm-c-update}]
    {\Gamma \vdash [l_i = \varsigma(x : A) . b_i , r_j = c_j].r_k \Leftarrow c :
    [l_i : B_i]}
    {\Gamma \vdash [l_i = \varsigma(x : A) . b_i , r_j = c_j] : [l_i : B_i]}
  \end{gather*}
  \end{spreadlines}
  \caption{Typing Judgments in $\textbf{O}_{ct}$}
  \label{Oct:typing}
\end{figure*}

Few comments on typing rules...\\

\begin{lem}\label{L:bot}
  For arbitrary object $o = [l_i = \varsigma(x : A) . b_i , r_i = c_i]$, if
  field $l_k$ has method $\varsigma(x : A) . (x.l_k)$, then the field $l_k$ may
  be typed by any type $B$. This will be the analogous notion of $Falsum$ (i.e.
  bottom ``$\bot$'') in our calculus.
\end{lem}

\begin{proof}
  Without loss of generality, suppose our object only has one method field named
  $l$. Let $B$ be an arbitrary type, let $A = [l : B]$,
  \[
    \infer[\textsc{tm-obj}]
    {\Gamma \vdash [l = \varsigma(x : A) . (x.l)] : [l : B]}
    {\infer[\textsc{tm-m-invoke}]
      {\Gamma , x : [l : B] \vdash x.l : B}
      {\infer[\textsc{tm-var}]
        {\Gamma , x : [l : B] \vdash x : [l : B]}
        {}
      }
    }
  \]
\end{proof}


In order for our calculus to even make sense (in a sense that typing rules
interact well with other judgments), we need to prove several important
property. One of which is to show our constraints satisfaction judgment as we
defined in Section 2 preserves the typing judgment. Maybe we can call this
subject satisfaction (in reference to subject reduction).\\

Before we going into proof of the theorem, we define another syntactic apparatus, that is, the
typing judgment for a set of objects. Let $S$ be a set of objects, $A$ is a
type, we write $\Gamma \vdash S : A$ to mean for all $o \in S$, we have $\Gamma
\vdash o : A$. This theorem is proved below:

\begin{lem}[substitution preserves typing]\label{L:subst-typing}
  Suppose our calculus was parametrized over algebra $\mathcal{A}$ and theory
  $\mathcal{T}$. If $\Gamma , x : A \vdash t : T$ and $\ectxt \vdash v : A$,
  then $\Gamma \vdash [v/x]t : T$.
\end{lem}

\begin{proof}
  Induction on the structure of term $t$: we will only prove the case for
  primitives here, rest of the cases all follows directly by the definition of
  the substitution function and inductive hypotheses.
  \begin{enumerate}
    \item $t = t_1(a_1 , \ldots , a_n)$ for primitive term, since we have $\Gamma , x : A \vdash
      t_1(a_1, \ldots , a_n) : T$, it must be the case that $T = \mathcal{A}$
      and have premises:
      \[
        t \in \mathscr{F}
        \quad \text{and} \quad
        \Gamma , x : A \vdash a_i : \mathcal{A} \text{ for all } i \in [n]
      \]
      By IH, we have $\Gamma \vdash [v/x]a_i : \mathcal{A}$ for all $i \in [n]$.
      Since by the definition of substitution, we have
      \[
        [v/x](t_1(a_1 \ldots , a_n)) = t_1([v/x]a_1 , \ldots , [v/x]a_n)
      \]
      therefore we have
      \[
        \infer[\textsc{tm-atomic}]
        {\Gamma \vdash t_1([v/x]a_1 , \ldots , [v/x]a_n) : T}
        {t_1 \in \mathscr{F}
          & \Gamma \vdash [v/x]a_i : \mathcal{A} \quad \forall i \in [n]}
      \]
  \end{enumerate}
\end{proof}

\begin{thm}[subject satisfaction \& subject reduction]
  Suppose our calculus was parametrized over algebra $\mathcal{A}$ and theory
  $\mathcal{T}$.
  \begin{enumerate}
    \item If $o(\iota(x) . e) \models_{\solver} S$, $\Gamma \vdash e :
      Id(\mathcal{A})$ and $\Gamma \vdash o : A$, then $\Gamma \vdash S : A$.
    \item If $\Gamma \vdash e : A$ and $e \twoheadrightarrow e'$, then $\Gamma
      \vdash e' : A$.
  \end{enumerate}
\end{thm}

\begin{proof}
  Mutually induction on the derivation of $o(\iota(x) . e) \models_{\solver} S$
  and $\Gamma \vdash e : A$.
  \begin{enumerate}
    \item \textsc{sat-eq}, then $S = \{o\}$, $\Gamma \vdash o : A$ by assumption.
    \item \textsc{sat-empty}, then $S = \varnothing$, this case is vacuously
      true.
    \item $\textsc{sat-update}_1$, then $e = a_1 \sim a_2$, $S = \{o.l
      \Leftarrow \varsigma(z).v_1\}$, $[o/x]a_1 = o.l$, where $l$ is a label in
      $o$, since $o$ was well-typed, suppose the field $l$ has type $B$. Then it
      suffices to show $\Gamma , z : A \vdash v_1 : B$. Since the equation $a_1
      \sim a_2$ well typed with type $Id(\mathcal{A})$, we must have $\Gamma
      \vdash a_1 : \mathcal{A}$ and $\Gamma \vdash a_2 : \mathcal{A}$. By
      \Cref{L:subst-typing} we have both $\Gamma \vdash o.l : \mathcal{A}$ and
      $\Gamma \vdash [o/x]a_2 : \mathcal{A}$. By the subject reduction mutual
      inductive hypothesis, we have $\Gamma \vdash v_2 : \mathcal{A}$. By
      transitivity of $\sim$-relation, we have $v_1 : \mathcal{A}$ as wanted.
      % \begin{align*}
      %   v_1 \sim 
      % \end{align*}
      % with $o.l = [o/x]a_1$. Induction on
      % the judgment $\Gamma \vdash e : Id(\mathcal{A})$ together with the fact
      % that $o.l = [o/x]a_i$ gives us only one case to consider:
      % $\textsc{id-var}_1$. Then we have $\Gamma \vdash a_2 \in \mathcal{A}$, thus
      % $\Gamma \vdash [o/x]a_2 : \mathcal{A}$. Therefore, applying
      % \textsc{tm-m-update} gives us the conclusion $\Gamma \vdash o.l \Leftarrow
      % \varsigma(z) . [o/x]a_2 : A$.
  \end{enumerate}
\end{proof}

\subsection{Subtyping}

\subsection{Inheritance}

\subsection{Constraint Dispatch}